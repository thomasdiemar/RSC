\documentclass[12pt]{article}

% ========== XeLaTeX Setup ==========
\usepackage{fontspec}
\setmainfont{Times New Roman}

\usepackage{geometry}
\geometry{margin=1in}

\usepackage{amsmath,amssymb,amsfonts}
\usepackage{enumitem}

\title{A Unified Thruster Configuration for Pure Translation and Pure Rotation\\
with Automatic Center-of-Mass Compensation}
\author{}
\date{}

\begin{document}
\maketitle

\section*{1. Goal}

We seek a \emph{single, fixed thruster configuration} mounted on a rigid body that can:

\begin{enumerate}[label=(\alph*)]
    \item Translate in any direction with \textbf{zero torque},
    \item Generate torque about any axis with \textbf{zero net force},
    \item Continue to achieve (a--b) even when the center of mass (CoM) \textbf{moves inside the vehicle}.
\end{enumerate}

The solution is to use a combination of:
\begin{itemize}
    \item 6 \textbf{radial thrusters} (pure force, zero torque),
    \item 6 \textbf{tangential thrusters} arranged in pairs (pure torque, zero net force).
\end{itemize}

At runtime, the CoM motion is compensated mathematically by updating the moment arms for each thruster.

\section*{2. Thruster Geometry in Body Frame}

All thrusters are mounted in a \textbf{body-fixed} frame $B$. Their positions and directions never change.

Let $L>0$ be a fixed distance from the origin of the body frame.

\subsection*{2.1 Radial Thrusters (T1--T6): Pure Force, No Torque}

\[
\begin{aligned}
\text{T1: } & r_1^B = ( L, 0, 0), & a_1^B = ( 1, 0, 0),\\
\text{T2: } & r_2^B = (-L, 0, 0), & a_2^B = (-1, 0, 0),\\[4pt]
\text{T3: } & r_3^B = ( 0, L, 0), & a_3^B = ( 0, 1, 0),\\
\text{T4: } & r_4^B = ( 0,-L, 0), & a_4^B = ( 0,-1, 0),\\[4pt]
\text{T5: } & r_5^B = ( 0, 0, L), & a_5^B = ( 0, 0, 1),\\
\text{T6: } & r_6^B = ( 0, 0,-L), & a_6^B = ( 0, 0,-1).
\end{aligned}
\]

These thrusters produce pure forces because $r_i^B \parallel a_i^B$.

\subsection*{2.2 Torque Thrusters (T7--T12): Pure Torque, Zero Net Force}

Three orthogonal torque pairs:

\[
\begin{aligned}
\text{T7: }  & r_7^B = (0, L, 0),  & a_7^B   = (0, 0,  1),\\
\text{T8: }  & r_8^B = (0,-L, 0),  & a_8^B   = (0, 0, -1),\\[4pt]
\text{T9: }  & r_9^B = (0, 0, L),  & a_9^B   = ( 1, 0, 0),\\
\text{T10: } & r_{10}^B = (0,0,-L), & a_{10}^B = (-1, 0, 0),\\[4pt]
\text{T11: } & r_{11}^B = ( L,0,0), & a_{11}^B = (0, 1, 0),\\
\text{T12: } & r_{12}^B = (-L,0,0), & a_{12}^B = (0,-1, 0).
\end{aligned}
\]

Firing each pair with equal magnitude yields zero net force but nonzero torque.

\section*{3. CoM-Dependent Moment Arms}

Let $c^B \in \mathbb{R}^3$ be the \textbf{current center of mass in the body frame}.  
This point can move as fuel is consumed or loads shift.

The moment arm for thruster $i$ relative to the CoM is:
\[
\tilde r_i = r_i^B - c^B.
\]

\section*{4. Force and Torque Equations}

Let $T_i \ge 0$ be the thrust magnitude of thruster $i$.

Force from thruster $i$:
\[
f_i = T_i\, a_i^B.
\]

Total force:
\[
F = \sum_{i=1}^{12} T_i\, a_i^B.
\]

Torque from thruster $i$:
\[
\tau_i = \tilde r_i \times (T_i\, a_i^B).
\]

Total torque:
\[
\tau = \sum_{i=1}^{12} T_i \left( \tilde r_i \times a_i^B \right).
\]

\section*{5. Wrench Matrix Representation}

Define
\[
A = [a_1^B\ \dots\ a_{12}^B] \in \mathbb{R}^{3\times12},
\]
\[
B(c^B) = [\tilde r_1\times a_1^B\ \dots\ \tilde r_{12}\times a_{12}^B] \in \mathbb{R}^{3\times12}.
\]

Stack thrust vector:
\[
T = (T_1,\dots,T_{12})^\top.
\]

Then:
\[
\begin{bmatrix} F \\ \tau \end{bmatrix}
=
W(c^B)\,T,
\qquad
W(c^B) =
\begin{bmatrix}
A\\[4pt]
B(c^B)
\end{bmatrix}.
\]

If $W(c^B)$ has rank $6$, we can generate arbitrary force+torque commands.

\section*{6. Mode 1: Pure Translation (Zero Torque)}

Given a desired force $F_{\text{des}}$ and zero torque:
\[
\tau_{\text{des}} = 0,
\]
solve:
\[
W(c^B)\,T =
\begin{bmatrix}
F_{\text{des}}\\[4pt]
0
\end{bmatrix},
\qquad 0 \le T_i \le T_{\max,i}.
\]

This yields translation in any direction with zero torque, even when CoM shifts.

\section*{7. Mode 2: Pure Rotation (Zero Net Force)}

Given desired torque $\tau_{\text{des}}$ and zero force:
\[
F_{\text{des}} = 0,
\]
solve:
\[
W(c^B)\,T =
\begin{bmatrix}
0\\[4pt]
\tau_{\text{des}}
\end{bmatrix},
\qquad 0 \le T_i \le T_{\max,i}.
\]

This produces rotation about any axis with zero translation for the current CoM.

\section*{8. CoM Adaptation}

When mass shifts:

\begin{enumerate}[label=(\arabic*)]
\item Measure or estimate new CoM $c^B$.
\item Recompute moment arms:
\[
\tilde r_i = r_i^B - c^B.
\]
\item Recompute torque matrix $B(c^B)$.
\item Resolve the force/torque allocation problem for the new CoM.
\end{enumerate}

No hardware changes are required.

\section*{9. Summary}

This single 12-thruster configuration:

\begin{itemize}[leftmargin=*]
    \item Provides full 3D translation: any $F = (F_x,F_y,F_z)$ with $0$ torque,
    \item Provides full 3D rotation: any $\tau = (\tau_x,\tau_y,\tau_z)$ with $0$ force,
    \item Works for any CoM location inside the body by updating the moment arms,
    \item Enables general 6-DOF control by solving $W(c^B)T = (F,\tau)$.
\end{itemize}

\end{document}
