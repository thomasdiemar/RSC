\documentclass[11pt]{article}

\usepackage[a4paper,margin=1in]{geometry}
\usepackage{amsmath,amssymb,mathtools}
\usepackage{array,booktabs,multirow}
\usepackage{enumitem}
\usepackage{hyperref}
\usepackage{bm}

\setlist[itemize]{topsep=2pt,itemsep=2pt}
\setlist[enumerate]{topsep=4pt,itemsep=4pt}

\newcommand{\LB}{\mathrm{LB}}
\newcommand{\UB}{\mathrm{UB}}
\newcommand{\B}{\mathrm{B}}
\newcommand{\ve}{x_e}
\newcommand{\xb}{\bm{x}_B}
\newcommand{\Be}{d_e}
\newcommand{\rc}{r}
\newcommand{\A}{A}
\newcommand{\Bmat}{B}
\newcommand{\uvec}{\bm{u}}
\newcommand{\lvec}{\bm{\ell}}
\newcommand{\thet}{\theta}
\newcommand{\thb}{\theta_{\mathrm{bound}}}

\begin{document}

\begin{center}
{\LARGE Bounded Preemptive Simplex (Lexicographic Goal Programming Core)}\\[2mm]
{\large How bounds are kept in the tableau, how LB/UB/B statuses work, how entering is chosen, and how bound hits (no pivot) differ from row pivots.}
\end{center}

\section*{1.\;Problem structure (generic)}

Variables $x_1,\dots,x_n$ have true physical bounds
\[
\ell_i \;\le\; x_i \;\le\; u_i \qquad (i=1,\dots,n).
\]
If a variable is nonbasic, it sits exactly at one bound:
\[
x_i=\ell_i\;(\LB)\quad\text{or}\quad x_i=u_i\;(\UB).
\]
If a variable is basic ($\B$), it can lie strictly between its bounds:
\[
\ell_i < x_i < u_i.
\]
Bounds are kept \emph{inside} the simplex (no substitutions like $x=\ell+\tilde{x}$).

\section*{2.\;Generic bounded simplex tableau}

A canonical tableau (illustrative):
\[
\begin{array}{c|c|rrrr|c}
\text{Basic} & \text{RHS} & x_1 & x_2 & x_3 & x_4 & \text{Status} \\
\hline
x_3 & 2 & +1 & -2 & 1 & 0 & \B \\
x_4 & 1 & 0 & +1 & -1 & 1 & \B \\
\hline
z\ (\max) & 7 & +3 & -1 & 0 & 0 & 
\end{array}
\]
Separately store nonbasic variables and their bound statuses:
\[
\begin{array}{c|c|c}
\text{Variable} & \text{Value} & \text{Status} \\
\hline
x_1 & x_1=\ell_1 & \LB \\
x_2 & x_2=u_2 & \UB
\end{array}
\]
In the bounded simplex, nonbasics sit at their \emph{real} bounds (not at $0$).

\section*{3.\;Reduced-cost eligibility (choosing the entering variable)}

For a \emph{maximization}:
\[
\begin{aligned}
&\text{Nonbasic at } \LB: && \text{may enter by increasing if } r_j>0.\\
&\text{Nonbasic at } \UB: && \text{may enter by decreasing if } r_j<0.
\end{aligned}
\]
For a \emph{minimization}, flip signs:
\[
\begin{aligned}
&\text{Nonbasic at } \LB: && \text{may enter if } r_j<0.\\
&\text{Nonbasic at } \UB: && \text{may enter if } r_j>0.
\end{aligned}
\]
If no nonbasic satisfies this, the current stage is optimal.

\section*{4.\;Augmented ratio test (bounded feature)}

Let $x_e$ be eligible. Define the attempted move direction:
\[
d_e=\begin{cases}
+1, & \text{if } x_e \text{ is at } \LB \text{ (increase)},\\
-1, & \text{if } x_e \text{ is at } \UB \text{ (decrease)}.
\end{cases}
\]
Let $\bm{d}=B^{-1}A_{:e}$ be the basic-direction column for $x_e$.
Movement with step $\Delta$ is
\[
\Delta x_e=d_e\,\Delta, \qquad \Delta \bm{x}_B=-\bm{d}\,\Delta.
\]

Two limits can stop us:

\paragraph{(A) Row limit: a basic hits LB or UB.}
For each basic $x_{B_i}$ with direction $d_i$:
\[
\theta_i=
\begin{cases}
\dfrac{x_{B_i}-\ell_{B_i}}{d_i}, & \text{if } d_e=+1,\ d_i>0, \\[1.1ex]
\dfrac{u_{B_i}-x_{B_i}}{-d_i}, & \text{if } d_e=+1,\ d_i<0.
\end{cases}
\]
If $d_e=-1$, LB/UB swap.

\paragraph{(B) Bound limit of the entering variable.}
\[
\theta_{\text{bound}}=
\begin{cases}
u_e - x_e, & \text{if } d_e=+1,\\
x_e - \ell_e, & \text{if } d_e=-1.
\end{cases}
\]

\paragraph{Step size.}
\[
\theta=\min\{\theta_{\text{bound}},\ \min_i \theta_i\}.
\]

\section*{5.\;Two possible outcomes}

\subsection*{(1) Strict bound hit of entering ($\theta=\theta_{\text{bound}}$)}
\begin{itemize}
\item \textbf{Preemptive bound hit:} no pivot.
\item Move $x_e$ to the opposite bound, update basics: $\Delta\bm{x}_B=-\bm{d}\,\theta$.
\item $x_e$ stays nonbasic, flips $\LB \leftrightarrow \UB$.
\end{itemize}

\subsection*{(2) Row pivot ($\theta=\theta_i$)}
\begin{itemize}
\item Perform a Gauss--Jordan pivot.
\item Entering becomes basic; leaving hits its bound and becomes $\LB$ or $\UB$.
\end{itemize}

\section*{6.\;Example iteration}

Initial:
\[
\begin{array}{c|c|c}
\text{var} & \text{value} & \text{status}\\\hline
x_1 & 0 & \LB\\
x_2 & 5 & \UB\\
x_3 & 2 & \B\\
x_4 & 1 & \B
\end{array}
\]

Maximization reduced costs:
\[
r_1=+3>0 \Rightarrow x_1\ \text{eligible from }\LB,\qquad
r_2=-1<0 \Rightarrow x_2\ \text{eligible from }\UB.
\]
Pick $x_1$. Suppose $\theta_{\text{bound}}=10$ and row $x_3$ reaches LB at $\theta=2$. Since $2<10$, pivot.

\[
\begin{array}{c|c|rrrr|c}
\text{Basic} & \text{RHS} & x_1 & x_2 & x_3 & x_4 & \text{Status} \\
\hline
x_1 & 2 & 1 & -2 & 1 & 0 & \B \\
x_4 & 1 & 0 & +1 & -1 & 1 & \B \\
\hline
z & 13 & 0 & +5 & -3 & 0 &
\end{array}
\]
Leaving $x_3$ hits LB, so $x_3\to \LB$.

If instead $\theta_{\text{bound}}=1<3$, then preemptive bound hit. $x_1$ moves to $u_1$, no pivot, status flips to $\UB$.

\section*{7.\;Degeneracy}

If $\theta=0$:
\begin{itemize}
\item If row-limited: pivot.
\item If bound-limited: flip $\LB \leftrightarrow \UB$.
\item Use Bland's rule to avoid cycling.
\end{itemize}

\section*{8.\;Difference from ordinary simplex}

\[
\begin{array}{l|l}
\text{Standard simplex} & \text{Bounded preemptive simplex} \\\hline
\text{Nonbasics at } 0 & \text{Nonbasics at LB/UB}\\
\text{No opposite-bound hits} & \text{LB}\!\to\!\text{UB or UB}\!\to\!\text{LB possible}\\
\text{Movement ends in pivot} & \text{May end in bound hit with no pivot}\\
\text{Needs variable substitution for bounds} & \text{Bounds handled directly}
\end{array}
\]

\section*{9.\;Summary}

\[
\begin{array}{l|l|l|l}
\text{Event} & \text{Stops movement?} & \text{Action} & \text{Status change}\\\hline
\text{Bound hit (strict)} & \text{Opposite bound first} & \text{No pivot} & \LB \leftrightarrow \UB\\
\text{Row pivot} & \text{Basic hits bound first (or tie)} & \text{Pivot} & \text{Enter }\to \B,\ \text{leave }\to \LB/\UB\\
\text{Degenerate }(\theta=0) & \text{Row or bound} & \text{Pivot or flip}; \text{Bland's} & \text{As above}\\
\text{No eligible } r_j & \text{---} & \text{Optimal} & \text{---}
\end{array}
\]

\vspace{1ex}
\noindent\emph{This is the mechanism used internally by lexicographic (preemptive) goal programming.}

\end{document}
