\documentclass{article}
\begin{document}
  Hello World! http://www.personal.ceu.hu/tex/math.htm
$ \alpha \beta \gamma $
$ \frac{x}{y} $
$  \int a^b $

Convert the LP to the following form:
Convert the minimization problem into a maximization one (by multiplying the objective function by -1).
All variables must be non-negative.
All RHS values must be non-negative (multiply both sides by -1, if needed).
All constraints must be in £ form (except the non-negativity conditions). No strictly equality or ³ constraints are allowed.
If this condition cannot be satisfied, then use the Initialization of the Simplex Method: Articicial-Free.

Convert all £ constraints to equalities by adding a different slack variable for each one of them.
Construct the initial simplex tableau with all slack variables in the BVS. The last row in the table contains the coefficient of the objective function (row Cj).

Determine whether the current tableau is optimal. That is:
If all RHS values are non-negative (called, the feasibility condition)
If all elements of the last row, that is Cj row, are non-positive (called, the optimality condition).

If the answers to both of these two questions are Yes, then stop. The current tableau contains an optimal solution.
Otherwise, go to the next step.

If the current BVS is not optimal, determine, which nonbasic variable should become a basic variable and, which basic variable should become a nonbasic variable. To find the new BVS with the better objective function value, perform the following tasks:
Identify the entering variable: The entering variable is the one with the largest positive Cj value (In case of a tie, we select the variable that corresponds to the leftmost of the columns) .
Identify the outgoing variable: The outgoing variable is the one with smallest non-negative column ratio (to find the column ratios, divide the RHS column by the entering variable column, wherever possible). In case of a tie we select the variable that corresponds to the upmost of the tied rows.
Generate the new tableau: Perform the Gauss-Jordan pivoting operation to convert the entering column to an identity column vector (including the element in the Cj row).
Go to step 4.



\end{document} 