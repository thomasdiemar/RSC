\documentclass[11pt]{article}

\usepackage{amsmath, amssymb}
\usepackage{geometry}
\usepackage{fontspec}

\setmainfont{Latin Modern Roman}
\geometry{margin=2.5cm}

\begin{document}

\section*{Bounded Preemptive Goal Programming for a 6--Thruster System}

We consider six thrusters with positions
\[
\begin{aligned}
r_1 &= (-1,-1,-1), & a_1 &= (0,0, 1),\\
r_2 &= ( 1,-1,-1), & a_2 &= (0,0, 1),\\
r_3 &= ( 0, 1,-1), & a_3 &= (0,0, 1),\\
r_4 &= (-1,-1, 1), & a_4 &= (0,0,-1),\\
r_5 &= ( 1,-1, 1), & a_5 &= (0,0,-1),\\
r_6 &= ( 0, 1, 1), & a_6 &= (0,0,-1),
\end{aligned}
\]
and thrust magnitudes
\[
0 \le T_i \le 1,\quad i = 1,\dots,6.
\]

\subsection*{Net Force}

Since all thrusters act along $\pm z$,
\[
F_x = 0, \qquad F_y = 0,
\]
and
\[
F_z = T_1 + T_2 + T_3 - T_4 - T_5 - T_6.
\]

\subsection*{Net Torque}

The torque from thruster $i$ is
\[
\tau_i = r_i \times (T_i a_i).
\]
Using
\[
r\times(0,0,1) = (y,-x,0),\qquad
r\times(0,0,-1) = (-y,x,0),
\]
we obtain the total torque
\[
\tau = (\tau_x,\tau_y,\tau_z)
\]
with
\begin{align*}
\tau_x &= -T_1 - T_2 + T_3 + T_4 + T_5 - T_6,\\
\tau_y &= \phantom{-}T_1 - T_2 - T_4 + T_5,\\
\tau_z &= 0.
\end{align*}

\subsection*{Goals and Priorities}

We use bounded preemptive goal programming with the following priorities:
\begin{enumerate}
  \item \textbf{Priority 1:} Eliminate torque:
  \[
    \tau_x = 0,\qquad \tau_y = 0.
  \]
  \item \textbf{Priority 2:} Maximize the vertical force $F_z$.
  \item \textbf{Priority 3:} Minimize $F_x$ and $F_y$ (already identically zero).
\end{enumerate}

Under the torque constraints and bounds $0 \le T_i \le 1$, we can solve for
\[
T_1 = T_2 + T_4 - T_5,\qquad
T_3 = 2T_2 - 2T_5 + T_6,
\]
and one finds that
\[
F_z = 4\,(T_2 - T_5), \qquad T_2 - T_5 \le \frac{1}{2},
\]
so the maximum achievable vertical force with zero torque is
\[
F_z^{\max} = 2.
\]

\subsection*{One Optimal Solution}

One convenient optimal solution is
\[
(T_1,T_2,T_3,T_4,T_5,T_6) = (0.5,\; 0.5,\; 1,\; 0,\; 0,\; 0),
\]
which satisfies
\[
\begin{aligned}
F &= (F_x,F_y,F_z) = (0,0,2),\\[4pt]
\tau &= (\tau_x,\tau_y,\tau_z) = (0,0,0),
\end{aligned}
\]
and respects the bounds $0 \le T_i \le 1$.

\end{document}
