\documentclass[12pt]{article}

% ========== XeLaTeX setup ==========
\usepackage{fontspec}
\setmainfont{Times New Roman}

\usepackage{geometry}
\geometry{margin=1in}

\usepackage{amsmath,amssymb,amsfonts}
\usepackage{enumitem}

\title{Bounded Preemptive Goal Programming\\for a 12-Thruster Configuration}
\author{}
\date{}

\begin{document}
\maketitle

\section*{1. Thruster Configuration (CoM at Origin)}

We consider 12 thrusters mounted on a rigid body with center of mass at the origin.
Each thruster $i$ has position $r_i \in \mathbb{R}^3$, direction $a_i \in \mathbb{R}^3$,
and thrust magnitude $T_i \in [0,1]$.

\subsection*{1.1 Radial Thrusters (T1--T6)}

\[
\begin{aligned}
\text{T1: } & r_1 = ( L, 0, 0), & a_1 = ( 1, 0, 0),\\
\text{T2: } & r_2 = (-L, 0, 0), & a_2 = (-1, 0, 0),\\[3pt]
\text{T3: } & r_3 = ( 0, L, 0), & a_3 = ( 0, 1, 0),\\
\text{T4: } & r_4 = ( 0,-L, 0), & a_4 = ( 0,-1, 0),\\[3pt]
\text{T5: } & r_5 = ( 0, 0, L), & a_5 = ( 0, 0, 1),\\
\text{T6: } & r_6 = ( 0, 0,-L), & a_6 = ( 0, 0,-1).
\end{aligned}
\]

These are radial: $r_i \parallel a_i$, so individually they produce zero torque.

\subsection*{1.2 Tangential Thrusters (T7--T12)}

\[
\begin{aligned}
\text{T7: }  & r_7   = (0, L, 0),  & a_7   = (0, 0,  1),\\
\text{T8: }  & r_8   = (0,-L, 0),  & a_8   = (0, 0, -1),\\[3pt]
\text{T9: }  & r_9   = (0, 0, L),  & a_9   = ( 1, 0, 0),\\
\text{T10: } & r_{10} = (0, 0,-L), & a_{10} = (-1, 0, 0),\\[3pt]
\text{T11: } & r_{11} = ( L, 0, 0), & a_{11} = (0, 1, 0),\\
\text{T12: } & r_{12} = (-L, 0, 0), & a_{12} = (0,-1, 0).
\end{aligned}
\]

\section*{2. Force and Torque Equations}

Let $T_i \in [0,1]$ be the thrust levels.  
The total force is
\[
F = (F_x,F_y,F_z) = \sum_{i=1}^{12} T_i a_i.
\]

By inspection,
\[
\begin{aligned}
F_x &= T_1 - T_2 + T_9 - T_{10},\\
F_y &= T_3 - T_4 + T_{11} - T_{12},\\
F_z &= T_5 - T_6 + T_7 - T_8.
\end{aligned}
\]

The torque from thruster $i$ is $\tau_i = r_i \times (T_i a_i)$, and total torque
$\tau = (\tau_x,\tau_y,\tau_z)$ is
\[
\tau = \sum_{i=1}^{12} r_i \times (T_i a_i).
\]

Radial thrusters T1--T6 generate no torque (since $r_i \times a_i = 0$), while:

\[
\begin{aligned}
r_7 \times a_7   &= (L,0,0), & r_8 \times a_8   &= (L,0,0),\\
r_9 \times a_9   &= (0,L,0), & r_{10} \times a_{10} &= (0,L,0),\\
r_{11} \times a_{11} &= (0,0,L), & r_{12} \times a_{12} &= (0,0,L).
\end{aligned}
\]

Hence
\[
\begin{aligned}
\tau_x &= L(T_7 + T_8),\\
\tau_y &= L(T_9 + T_{10}),\\
\tau_z &= L(T_{11} + T_{12}).
\end{aligned}
\]

\section*{3. Bounded Preemptive Goal Programming}

We impose bounds
\[
0 \le T_i \le 1,\quad i=1,\dots,12,
\]
and use a preemptive (lexicographic) priority structure:

\begin{enumerate}[label=\textbf{P\arabic*:}]
    \item Maximize $\tau_x$ (x-axis torque),
    \item Minimize other torque components $|\tau_y|,|\tau_z|$,
    \item Treat force as last priority (e.g.\ minimize net force magnitude).
\end{enumerate}

Small nonnegative tolerances $\varepsilon,\varepsilon_y,\varepsilon_z$ are used
in achievement constraints to allow numerical slack.

\subsection*{3.1 Priority P1: Maximize $\tau_x$}

Solve the LP:
\[
\begin{aligned}
\max\ & \tau_x = L(T_7 + T_8)\\
\text{s.t. }&
\begin{cases}
\tau_y = L(T_9 + T_{10}),\\
\tau_z = L(T_{11} + T_{12}),\\[2pt]
F_x = T_1 - T_2 + T_9 - T_{10},\\
F_y = T_3 - T_4 + T_{11} - T_{12},\\
F_z = T_5 - T_6 + T_7 - T_8,\\[2pt]
0 \le T_i \le 1,\quad i=1,\dots,12.
\end{cases}
\end{aligned}
\]

Let the optimal value be $\tau_x^*$.  
We then add the \emph{achievement constraint} for all lower-priority levels:
\[
\tau_x \;\ge\; \tau_x^* - \varepsilon.
\]

\subsection*{3.2 Priority P2: Minimize Other Torque}

Introduce nonnegative magnitude variables
\[
\begin{aligned}
-u_{\tau_y} \le \tau_y \le u_{\tau_y}, &\quad u_{\tau_y} \ge 0,\\
-u_{\tau_z} \le \tau_z \le u_{\tau_z}, &\quad u_{\tau_z} \ge 0.
\end{aligned}
\]

Solve:
\[
\begin{aligned}
\min\ & u_{\tau_y} + u_{\tau_z}\\
\text{s.t. }&
\begin{cases}
\tau_x = L(T_7 + T_8),\\
\tau_y = L(T_9 + T_{10}),\\
\tau_z = L(T_{11} + T_{12}),\\[2pt]
F_x = T_1 - T_2 + T_9 - T_{10},\\
F_y = T_3 - T_4 + T_{11} - T_{12},\\
F_z = T_5 - T_6 + T_7 - T_8,\\[2pt]
-u_{\tau_y} \le \tau_y \le u_{\tau_y},\\
-u_{\tau_z} \le \tau_z \le u_{\tau_z},\\[2pt]
0 \le T_i \le 1,\\[2pt]
\text{Achievement from P1: } \tau_x \ge \tau_x^* - \varepsilon.
\end{cases}
\end{aligned}
\]

Let the optimal values be $u_{\tau_y}^*, u_{\tau_z}^*$.  
We then add \emph{achievement constraints} for P2:
\[
u_{\tau_y} \le u_{\tau_y}^* + \varepsilon_y,\qquad
u_{\tau_z} \le u_{\tau_z}^* + \varepsilon_z.
\]

\subsection*{3.3 Priority P3: Force as Last Priority}

Now we treat the net force as the lowest priority quantity.
Introduce magnitude variables:
\[
\begin{aligned}
-u_{Fx} \le F_x \le u_{Fx}, &\quad u_{Fx} \ge 0,\\
-u_{Fy} \le F_y \le u_{Fy}, &\quad u_{Fy} \ge 0,\\
-u_{Fz} \le F_z \le u_{Fz}, &\quad u_{Fz} \ge 0.
\end{aligned}
\]

Solve:
\[
\begin{aligned}
\min\ & u_{Fx} + u_{Fy} + u_{Fz}\\
\text{s.t. }&
\begin{cases}
F_x = T_1 - T_2 + T_9 - T_{10},\\
F_y = T_3 - T_4 + T_{11} - T_{12},\\
F_z = T_5 - T_6 + T_7 - T_8,\\[2pt]
\tau_x = L(T_7 + T_8),\\
\tau_y = L(T_9 + T_{10}),\\
\tau_z = L(T_{11} + T_{12}),\\[2pt]
-u_{Fx} \le F_x \le u_{Fx},\\
-u_{Fy} \le F_y \le u_{Fy},\\
-u_{Fz} \le F_z \le u_{Fz},\\[2pt]
0 \le T_i \le 1,\\[2pt]
\text{Achievement P1: } \tau_x \ge \tau_x^* - \varepsilon,\\
\text{Achievement P2: } u_{\tau_y} \le u_{\tau_y}^* + \varepsilon_y,\ 
                        u_{\tau_z} \le u_{\tau_z}^* + \varepsilon_z.
\end{cases}
\end{aligned}
\]

This finds thrust levels that:
\begin{itemize}[leftmargin=*]
    \item Preserve the \textbf{maximum} x-torque from P1,
    \item Preserve the \textbf{minimum} achievable other torques from P2,
    \item And, among those, \textbf{minimize net force} as the final priority.
\end{itemize}

\section*{4. Priority Summary}

In words, the bounded preemptive goal programming priorities are:
\begin{enumerate}[label=\textbf{P\arabic*:}]
    \item Maximize $\tau_x$ (primary torque objective),
    \item Minimize $|\tau_y|$ and $|\tau_z|$ (secondary torque cleanup),
    \item Minimize $|F_x|,|F_y|,|F_z|$ (force disturbance as last priority).
\end{enumerate}

\end{document}
